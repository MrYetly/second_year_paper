\documentclass{article}

\usepackage{amsmath}
\usepackage{amssymb}
\usepackage{amsthm}
\usepackage{rotating}

\newtheorem{proposition}{Proposition}
\newtheorem{definition}{Definition}
\DeclareMathOperator*{\argmin}{arg\,min}
\DeclareMathOperator*{\argmax}{arg\,max}
\newcommand\numberthis{\addtocounter{equation}{1}\tag{\theequation}}

\title{2nd year paper}
\author{Ivan Anich and Matt Van Essen}

\begin{document}
\maketitle


	
	The theory of public goods is concerned about what is just. Erik Lindahl's classic result is really more of an argument for how economic resources should be justly allocated in the presence of a public good. He argues that such an allocation should be Pareto optimal, and that individuals should be taxed according to the benefit they receive from the public good such that the sum of all taxes exactly covers the cost of providing it. His equilibrium is just because it is individually rational for individuals to accept the taxes administered to them and because it leaves those individuals as well off as they can be. It has become a standard by which theories of the provision of public goods are measured.
	
	Yet, it is inadequate because it is restricted to linear taxation. It cannot take into account the distribution of surplus created by the taxes it proscribes. As we show, this uncontrolled margin allows for outcomes that are very unjust. In order to build a more rigorous standard, we introduce the concept of a generalized Lindahl equilibrium.
	
	
	Our generalized Lindahl equilibrium extends the classical result. It is Pareto optimal, individually rational, and balances the public budget. It also requires that individuals receive personalized tax \textit{schedules} instead of a single personalized tax. 
	
	This additional margin allows new problems to be solved. Suppose there is a public good present in an economy, and that a social welfare function represents the preferences of a society concerning the distribution of surplus in that economy. Furthermore, suppose this society wants to optimizes the distribution of surplus according to this social welfare function, provide a Pareto optimal allocation of goods, and finance the provision of the public good with taxes that are individually rational to accept. The outcome that achieves all of this is a generalized Lindahl equilibrium. We define the general form of this problem, and solve an example of it with a CES welfare function. We then propose a mechanism that implements generalized Lindahl equilibria when such a society has incomplete information. This mechanism is inspired by Walker's (1981) mechanism for attaining classical Lindahl allocations and so is called a generalized Walker mechanism. We show that the Nash equilibria of games induced by our generalized Walker mechanism are generalized Lindahl equilibria. 
	
	
	Our work traces its history back to that of Lindahl (1919 [1958) and Wicksell (1896 [1958). Lindahl asked, when two parties have equal political power but different valuations of a public good, what level of the public good should be provided and how should it be paid for? He argued that each share of costs should be determined by each party’s marginal valuation of the public good, and that the level provided should be that which maximizes the net gain of both parties. Wicksell pointed out that the definition of justice is itself a problem and argued for a simple solution, unanimity: in order to achieve economic justice, a society should unanimously choose both how public funds should be spent and how the burden of raising those funds should be shared. Like Lindahl, Wicksell  believed the burden to pay for a public good should be proportional to the benefit received from it.
			
	This paper contributes to work to extend Lindahl's original theory in order to address its shortcomings. Foley (1970) rigorously stated Lindahl's result and then studied how it related to core theory and showed that all Lindahl equilibria are in the core of an economy.\footnote{The core of an economy is the set of outcomes that optimizing agents converge to in a game theoretic setting in which said agents have the ability to form, and strategize as, coalitions.} Kaneko (1977) pointed out that Lindahl's equilibrium does not dictate how profits should be redistributed, and so in the presence of profits (i.e. non-linear costs) not all Lindahl equilibria are in the core of the economy. He created a version of Lindahl's equilibrium that avoided this problem: he introduced the concept of a ratio equilibrium, in which the total costs of production of public goods are distributed among consumers according to a vector of ratios. These ratios define taxes paid to firms such that the costs of provision are exactly covered and no firm earns profits. The problem of how positive profits should be redistributed would not be addressed until Mas-Colell and Silvestre (1989). Their theory of cost-share equilibrium endogenously determines profit shares for a public firm, and shows that profits should be distributed according to the benefit an individual receives from a public good. This result built on Mas-Colell's (1980) previous work, in which he showed when costs do not have a linear structure Lindahl's original result is still efficient and able to be attained in a decentralized manner. Diamantaras and Gilles (1996) extended this result by showing it continues to hold in the presence of an arbitrary number of private goods.  
	
	Our generalized Walker mechanism contributes to the study of mechanism design in the presence of public goods. The mechanisms of Groves and Ledyard (1977), Hurwicz (1979), and Walker (1981) all provided solutions to the free rider problem. Groves and Ledyard's mechanism is incentive compatible and was shown to implement the first and second theorems of welfare economics, but it is not always individually rational. Walker's mechanism addressed this issue by designing a mechanism whose outcomes were always Lindahl allocations, and so individually rational by definition. Varian (1994a) provides a mechanism inspired by Coase-ian negotiation that addresses the problem of how to set a Pigouvian tax when net social benefits are unknown. Varian (1994b) then goes on to show that when contributions to a public good are made sequentially, total contributions are never more than they are when made simultaneously, and introduces a mechanism capable of implementing Lindahl allocations in a sequential-move game.
	
	As we will discuss in section 5, our theory also has the potential to contribute to more recent work on mechanisms with stable equilibria, in particular in the presence of public goods.\footnote{In an experimental setting, participants often do not engage in Nash equilibrium play. Participants instead begin out of equilibrium, and adjust their behavior as they play the same game repeatedly. Theory suggests that if the equilibria of such a game are stable, players' responses should move toward a Nash equilibrium as they learn. Experimental evidence supports this (Chen 2002).} Milgrom and Roberts (1990) derive a way to compute the upper and lower bounds of the set of undominated strategies for supermodular games. If the set of undominated strategies is a singleton, then they argued that out of equilibrium play should eventually converge to a single Nash equilibrium, and that this convergence is robust to a number of different assumptions about learning. Mathevet (2010) attempts to use supermodular games to solve the dilemma that a mechanism can have either a simple structure or a single equilibrium. He introduces a method of mechanism design in which the interval between Nash equilibria is minimized. His method creates mechanisms robust to bounded rationality (i.e. out of equilibrium play). Healy and Mathevet (2012) go on to show that supermodularity is not always a sufficient condition for stability of equilibria. They introduce contraction mechanisms. Such mechanisms lead to best response functions that are contraction mappings, which sequentially reduce the set of undominated strategies and so lead to stable equilibria. Chen (2002) invents a family of supermodular mechanisms that achieves Lindahl allocations. However, Van Esson (2013) shows that the supermodularity of Chen's mechanism is not a sufficient condition for stability. In the spirit of Chen and Healy and Mathevet, Van Essen creates a supermodular and contractionary mechanism that leads to stable Lindahl equilibria. 
	
	The rest of this paper is structured as follows. Section 1 introduces an economic setting and describes a classical Lindahl equilibrium in this setting. Section 2 introduces the concept of generalized Lindahl equilibrium and shows how the concept can be used to solve a welfare maximization problem with a CES welfare function. Section 3 defines a generalized Walker mechanism, and proves that any Nash equilibrium it implements is a generalized Lindahl equilibrium. Section 4 discusses the direction of future research. Section 5 concludes.
	

\section{Setting and Classical Lindahl Equilibrium}

There will be \(i = 1, ..., n \geq 2\) individuals in the economy, each represented by a convex consumption set \(C_i = \mathbb{R}^2\) and an initial endowment of a private good \(\omega_i > 0\). They consume bundles of a public good \(x\) a private good \(y_i\), and in doing so gain utility according to utility functions of the following form.

\begin{equation}
u_i(x, y_i) = \int_0^x v_i(t) dt + y_i
\label{set_util}
\end{equation}

This functional form simplifies later analysis, as \(v_i\) is the the marginal rate of substitution between the public good and the private good for individual \(i\), and \(\frac{\partial u_i}{\partial x} = v_i(x) > 0\) and \(\frac{\partial^2 u_i}{\partial x^2} = v_i'(x) < 0\) for \(x \in \mathbb{R}_+\).

The public good is produced in a constant returns to scale production process that uses the private good as an input: for each unit of the public good produced, \(c\) of the private good must be sacrificed. If \(\Omega = \sum_i \omega_i\), then the most public good that can be produced is \(\frac{\Omega}{c}\). An allocation in this economy is then an \((n+1)\)-tuple \((x, y_1, ..., y_n) \in \mathbb{R}_+^{n+1}\). 

We shall assume that there is a unique Pareto optimal allocation in the economy, such that a unique level of the public good \(x^{PO}\) is Pareto optimal. This is to say that there is a level of the public good that satisfies the Samuelson marginal condition: \(\sum_i v_i(x^{PO}) = c\) for some \(x^{PO} \in [0, \frac{\Omega}{c}]\). To ensure this, we assume \(\sum_i v_i(0) > c\) and \(\sum_i v_i(\frac{\Omega}{c}) < c\). In addition, we shall also assume that the initial endowments of the private good are sufficiently large for feasible provision of \(x^{PO}\), \(\omega_i \geq v_i(0) x^{PO}\).

\subsection{Classical Lindahl Equilibrium} \label{CLE}

An unregulated market cannot efficiently provide a public good. A firm cannot earn revenue by selling the right to consume a public good because it cannot prevent the consumption of a public good by those who have not paid for it. Consumers face a strong incentive not pay for the provision of a public good because once the public good is provided, the consumption of it by one individual does not prevent the consumption of it by another. These are the usual non-rival and non-excludable characteristics of a public good. Some sort of market regulation is required to elicit and finance the efficient provision of the public good. Individuals can be taxed and production can be centralized by a government. One theoretical equilibrium that allows a Pareto optimal amount of the public good to be provided is a classical Lindahl equilibrium.

A classical Lindahl equilibrium uses individualized taxes, known as Lindahl prices, to raise revenue for the provision of a public good. It finds the socially optimal level of the public good and then charges each consumer their marginal valuation of the public good at that level. With constant marginal costs of provision and downward sloping demand curves, a classical Lindahl equilibrium ensures that it is individually rational for individuals to pay the tax levied upon them and that the sum of such taxes exactly covers the provision of the public good. 

For example, consider an instance of the economic setting outlined above with \(n=2\), \(c=5\), and the following utility functions for the two individuals 

\begin{align*}
u_1(x, y_1) &= \int_0^x \left( 20 - 2t \right) dt + y_1 \\
u_2(x, y_2) &= \int_0^x \left( 6 - 0.1t \right) dt + y_2
\end{align*}

\noindent And so

\begin{align*}
v_1(x) &= 20 - 2x \\
v_2(x) &= 6 - 0.1x
\end{align*}

\begin{figure}
\center{Classical Lindahl Equilibrium}
\includegraphics[width=\textwidth]{classical_lindahl_plots.pdf}
\caption{An example of a classical Lindahl Equilbrium for \(n=2\) consumers. The vertical grey dotted line is located at the Pareto optimal level of the public good. Consumer 1 is charged nothing, while consumer 2 is charged 5 units of the private good.}
\label{classical_lind}
\end{figure}

This setting is shown graphically in figure \ref{classical_lind}. To construct a classical Lindahl equilibrium we would first find the Pareto optimal level of the public good \(x^{PO}\) via the Samuelson marginal condition, \(\sum_i v_i(x^{PO}) = c\). Then, because the private good is a numeraire, the tax \(t_i\) levied upon each individual would be equal to their marginal rate of substitution evaluated at \(x^{PO}\): \(t_i = v_i(x^{PO})\). This tax is their Lindahl price. Here, the Pareto optimal amount of the public good is \(x^{PO} = 10\). The marginal benefit of the public good at this level is 0 for consumer 1 and 5 for consumer 2, and so \(t_1 = 0\) and \(t_2 = 5\). 

We can see that the classical Lindahl equilibrium leaves both consumers better off than they would be if no public good was provided, so the outcome is indeed individually rational. It also raises tax revenue that covers the cost of providing the public good. But, it is also clear that consumer 1 is left far better off than consumer 2, even though consumer 2 incurs all of the costs.

This illustrates how a Lindahl equilibrium does not take into account the distribution of surplus among consumers when optimizing the provision of a public good. Even though the outcome is individually rational, some consumers may be left far better off than others. In order to address this flaw, we introduce the concept of a generalized Lindahl equilibrium. 

\section{Generalized Lindahl Equilibrium}

A classical Lindahl equilibrium must be individually rational and raise tax revenue equal to the cost of provision of the public good. As stated in the previous section, the disregard of consumer surplus in such an equilibrium can lead to unfair outcomes. In this section, I explain how a generalization of the classical Lindahl equilibrium can allow a social planner to construct far more equitable outcomes. 

The unjustness of the classical result can be thought of as a result of the linearity of prices faced by consumers. Consumers are charged one personalized tax across all margins of consumption of the public good. The surplus they attain at each margin is then the difference between their marginal valuations and these constant taxes. If those taxes were non-linear, a more equitable outcome could be achieved. A tax schedule could be created such that the equity of the benefits accruing to each consumer were taken into account at each margin of provision of the public good.  

A generalized Lindahl equilibrium sees each consumer receive a personalized tax schedule, instead of a single personalized tax. The outcome is still Pareto optimal, individually rational, and covers the cost of provision, as in the classical case, but now a social planner can manipulate said tax schedules to distribute consumer surplus as they see fit. 

\begin{definition}
A profile of individualized price schedules induces a generalized Lindahl equilibrium if it is (1) individually rational for each individual to demand the Pareto optimal amount of the public good and (2) if the cost of providing the public good is exactly covered by taxes paid by each individual. 
\end{definition} 

Note that a generalized Lindahl equilibrium is agnostic about the specific distribution of consumer surplus. It merely requires that a tax schedule is administered. Consequently, there are many generalized Lindahl equilibria capable of providing the Pareto optimal level of the public good. The social planner can chose an outcome that obtains a desired distribution from among all feasible generalized Lindahl equilibria. This will be explored further in the next section.

For now, I shall use the economic setting described above to discuss some necessary conditions for a generalized Lindahl equilibrium. A personalized tax schedule is defined as a function \(p_i : \mathbb{R} \rightarrow \mathbb{R}\), such that the total tax revenue collected from individual \(i\) for \(x\) of the public good is is \(T_i(x) = \int_0^x p_i(t) dt\). The budget constraint of a customer is then \( \int_0^x p_i(t) dt + y_i = \omega_i \). If we use this to solve for \(y_i(x)\) and substitute this into the individual's utility function, we can express their utility in terms of their initial allocation and the level of public good provided.

\begin{equation}
u_i(x, y_i(x)) = u_i \left( x, \omega_i - \int_0^x p_i(t) dt \right) = \int_0^x \left[ v_i(t) - p_i(t) \right] dt + \omega_i
\label{gle_util}
\end{equation}

\noindent From this we get the fist of two necessary conditions for individual rationality: 

\begin{equation}
\left. \frac{\partial u_i(x, y_i(x))}{\partial x} \right|_{x=x^{PO}} = 0 \rightarrow v_i(x^{PO}) = p_i(x^{PO})
\label{gen_ir1}
\end{equation}

\noindent The prices charged at the margin for the public good in a generalized Lindahl equilibrium must be the Lindahl prices from the classical Lindahl equilibrium. The second condition for individual rationality requires that customers are indeed better off at the Pareto optimal outcome than they would be with zero public good provided.

\begin{equation}
u_i(x^{PO}, y_i(x^{PO})) \geq u_i(0, y_i(0))
\label{gen_ir2}
\end{equation}


\noindent And finally, we must have the tax revenues cover the cost of provision.

\begin{equation}
\sum_i  \int_0^{x^{PO}} p_i(t) dt = cx^{PO}
\label{gen_cost}
\end{equation}

 Equations \ref{gen_ir1}, \ref{gen_ir2}, and \ref{gen_cost} must hold for an equilibrium to qualify as a generalized Lindahl equilibrium. In the next section, they shall function as constraints on a social welfare problem facing a social planner who wants to optimally provide a public good, and who has CES preferences for the surpluses of those it oversees.


\subsection{Generalized Lindahl Equilibrium under a CES Social Welfare Function}

We can imagine a society's preferences for the distribution of surplus among its constituents being encoded into a marginal social welfare function. The maximization of the integral of this social welfare function, subject to the conditions necessary for a generalized Lindahl equilibrium stated in the previous section, is a theoretical representation of how such a society could provide a public good in an equitable and Pareto optimal way. 

In general, let's say that this marginal social welfare function is \(W : \mathbb{R}_+^n \rightarrow \mathbb{R}\). It is a function of the marginal consumer surpluses of each individual in a society \(v_i(t) - p_i(t)\) and is differentiable, quasi-concave, and increasing in each of its arguments, with \(W(\boldsymbol{0}) = 0\). We restrict \(W\) to the non-negative elements of \(\mathbb{R}^n\) to help ensure individual rationality; this restriction implies \(v_i(t) - p_i(t) \geq 0 ~ \forall ~ t \in [0, x^{PO}]\); additionally we require  \(v_i(t) - p_i(t) < 0 ~ \forall ~ t > x^{PO}\); in tandem these constraints provide a sufficient condition for individual rationality. The problem facing a society is then

\begin{align}
\max_{\{ p_i(t)\}_{i=1}^n} &\int_0^{x^{PO}} W(v_1(t) - p_1(t), ..., v_n(t) - p_n(t)) dt 
\\
s.t. ~ & v_i(x^{PO}) = p_i (x^{PO}) 
\\
& v_i(t) - p_i(t) \geq 0 &\forall ~t \in [0, x^{PO}] \label{cons_pos}
\\
& v_i(t) - p_i(t) < 0 &\forall ~t > x^{PO} \label{cons_neg}
\\
& \int_0^{x^{PO}} \sum_{i=1}^n p_i(x) dx = cx^{PO} 
\end{align}

We now solve a specific example of this problem in which society has CES preferences over individual marginal surpluses: 

\begin{align}
\max_{\{ p_i(t)\}_{i=1}^n} &\int_0^{x^{PO}} \left[ \sum_{i=1}^n \beta_i (v_i(x) - p_i(x))^\alpha \right]^{\frac{1}{\alpha}} dx \label{objective}
\\
s.t. ~ & v_i(x^{PO}) = p_i (x^{PO}) \label{margin_constraint}
\\
& \int_0^{x^{PO}} \sum_{i=1}^n p_i(x) dx = cx^{PO} \label{iso_constraint}
\end{align}

In this problem, \ref{cons_pos} and \ref{cons_neg} will not bind. \(\beta_i\) is an arbitrary weight for individual \(i\)'s surplus and \(-\infty > \alpha \leq 1\). Equation \ref{margin_constraint} ensures that the level of the public good provided is the level demanded by each individual, and equation \ref{iso_constraint} ensures that tax revenue will equal the cost of provision.  

This is a calculus of variations problem. Throughout its solution any dependency on \(x\) is suppressed. The necessary conditions to solve it are \ref{margin_constraint}, \ref{iso_constraint}, and \(n\) Euler equations, one for each individual:

\begin{equation}
\lambda = -\beta_j (v_j - p_j)^{\alpha - 1} \left[ \sum_{i=1}^n \beta_i (v_i - p_i)^\alpha \right]^{\frac{1- \alpha}{\alpha}} \quad \forall ~ j = 1, ..., n
\end{equation}

\noindent \(\lambda\) is the Lagrangian multiplier for the iso-perimetric constraint \ref{iso_constraint}. If we set the Euler equations for individuals 1 and 2 equal to each other and simplify, we get 

\[
\left( \frac{\beta_2}{\beta_1} \right)^\sigma (v_1 - p_1) = v_2 - p_2
\]

\noindent Where \(\sigma = \frac{1}{1 - \alpha}\) is the elasticity of substitution between the surplus of any two individuals. I will define \( \Omega_{21} = \left( \beta_2 / \beta_1 \right)^\sigma\). Substituting this and rearranging, we get

\[
p_2 - \Omega_{21} p_1 = v_2 - \Omega_{21}v_1
\]

\noindent If we perform a similar manipulation among all \(n\) Euler conditions, we are left with a system of \(n-1\) equations

\begin{align*}
p_2 - \Omega_{21} p_1 &= v_2 - \Omega_{21}v_1
\\
p_3 - \Omega_{32} p_2 &= v_3 - \Omega_{32}v_2
\\
&\vdots
\\
p_n - \Omega_{n,n-1} p_{n-1} &= v_n - \Omega_{n,n-1}v_{n-1}
\end{align*}

\noindent One way to ensure that the iso-perimetric constraint holds is to assume that the taxes levied at each margin are always equal to the marginal cost of production.

\begin{equation}
p_1 + p_2 + ... + p_n = c
\label{iso_simp}
\end{equation}

\noindent With this and the system of \(n-1\) equations above, we now have a system of \(n\) equations that we can use to solve for closed-form solutions of individualized price schedules (the full solution is detailed in the appendix). Those solutions are

\begin{equation}
p_j =  v_j - \frac{\beta_j\strut^\sigma}{\sum_{i=1}^n \beta_i\strut^\sigma} \left[  \sum_{i =1}^n (v_i)  - c \right]
\label{closed_p}
\end{equation}

In this solution, the optimal personalized tax schedule has two parts. Individuals are taxed at each margin according to their private valuation at that margin, \(v_j\). This baseline is then adjusted by what fraction of the marginal net social benefit (\( \sum_{i =1}^n (v_i)  - c\)) is redistributed back to them, as dictated by what share their social weights have in society (\(\beta_j\strut^\sigma / \sum_{i=1}^n \beta_i\strut^\sigma\)). The redistributive nature of the tax schedules is made clear when the marginal net benefit earned under them is stated:

\[
b_j \equiv v_j - p_j = \frac{\beta_j\strut^\sigma}{\sum_{i=1}^n \beta_i\strut^\sigma} \left[  \sum_{i =1}^n (v_i)  - c \right]
\]

\noindent Individuals are taxed such that all of their private surplus is taken away, and it is only the redistribution of net \textit{social} benefit that leads to any individual being better off in the end. 

For very small elasticities of substitution the complementarity of individuals' surpluses overrides their social weights. They are treated more or less equally. When \(\sigma = 0\), consumer surpluses are perfect complements and we are left with the following tax schedules.

\[
p_j = v_j - \frac{1}{n} \left[ \sum_{i =1}^n (v_i) - c \right]
\]

\noindent Individuals are left with equal shares of the net social benefit at each margin of provision of the public good. 

When \(\sigma=1\) the original marginal social welfare function becomes a Cobb-Douglas function and the personalized tax schedules become

\[
p_j = v_j - \frac{\beta_j}{\sum_{i=1}^n \beta_i} \left[  \sum_{i =1}^n (v_i) - c \right]
\]

\noindent The weights now have an effect because individuals' surpluses are substitutable. An individual with a higher social weight now receives a larger share of the net social benefit at each margin. 

As \(\sigma \rightarrow \infty\), individuals' surpluses become perfect substitutes. Those with the highest social weight (those indexed by \(j \in J_{max}\), where \( J_{max} \equiv \{j : \beta_j = \max \{\beta_i\}_1^n\}\)) receive all of the marginal net social benefit. Among these individuals, society is indifferent about the distribution of surplus. We shall assume that it is redistributed evenly among them.  Those with lower weights receive no redistribution. 

\[
p_j = \begin{cases}
v_j - \frac{1}{\left| J_{max} \right|} \left[ \sum_{i =1}^n (v_i) - c \right] & if ~ j \in J_{max} \\
v_j & else
\end{cases}
\]


Figure \ref{CES_plots} plots some examples of \ref{closed_p} for various values of \(\sigma\). We reuse the setting used to expose the unjustness of the classical Lindahl equilibrium in section \ref{CLE}: there are two consumers with \(v_1(x) = 20 - 2x\) and \(v_2(x) = 6 - 0.1x\), the marginal cost of providing the public good is 5, and a Pareto optimal amount of it is 10. \(\beta_1\) is 10 and \(\beta_2\) is 1. 

\begin{sidewaysfigure}
\center{Price Schedules, Marginal Benefits, and Marginal Cost in a 2 Consumer Setting for Various Values of \(\sigma\)}
\includegraphics[width = \textwidth]{lindahl_plots}
\caption{Above are several representations of price schedules leading to generalized Lindahl equilibria for a setting with two individuals and various values of \(\sigma\), the elasticity of substitution between the surpluses of any two individuals. The marginal benefit functions are \(v_1(x) = 20 - 2x\), \(v_2(x)=6 - x/10\). Social welfare weights are \(\beta_1 = 10\) and \(\beta_2 = 1\). Marginal cost of providing the public good is 5, and the Pareto optimal amount of it is 10.}
\label{CES_plots}
\end{sidewaysfigure}

\section{Generalized Walker Mechanism}

One of the strongest assumptions made in the previous section is that private valuations of the public good are known by the social planner. In an effort to relax this assumption, we now describe a mechanism a social planner could implement in order to obtain a generalized Lindahl equilibrium when they have no such knowledge. When this mechanism is implemented, it induces a generalized Lindahl equilibrium. We call it a generalized Walker mechanism in respect to the work of Walker (1981) from whom we draw inspiration.

A mechanism is defined by the messages agents are allowed to choose and the outcome functions that convert these messages into outcomes. In our environment (with a slight change of notation) an outcome is specified by a level of the public good \(x\) and a set of personalized tax schedules for each individual \(\tau_i\) such that each individual pays in total \(T_i(x) = \int_0^x \tau_i(t) dt\). The messages sent by individuals \(i=1, ..., n\) are continuous functions \(p_i : \mathbb{R} \rightarrow \mathbb{R}\). The list of all of these messages is a message profile \(\mathbf{p} = (p_1, ..., p_n)\). There are \(n+1\) outcome functions, one to determined the provision of the public good,  \(\mathcal{X} : \prod_i^n C(\mathbb{R}) \rightarrow \mathbb{R}\), and \(n\) to determined what tax schedules are to be administered, \(\mathcal{T}_i : \prod_i^n C(\mathbb{R}) \rightarrow C(\mathbb{R})\). The functional forms of these outcome functions are
	
	
\begin{align}
\mathcal{X}(\mathbf{p}) &= \argmax_{t \geq 0} \int_o^t \sum_{i=1}^n p_i(m) dm \label{oc_x} \\
\mathcal{T}_i(\mathbf{p}) &= \frac{c}{n} + p_{i+2} - p_{i+1} \label{oc_t} 
\end{align}	
	
\noindent Note that the subscript \(i + j\) is understood to be a modulo operation, \(n+j= j\), and \(c\) is once again the marginal cost of the public good. This follows the mechanism design in Walker (1981).

 This mechanism defines a game form: when preferences are added to it a game is induced. The outcome functions \ref{oc_x} and \ref{oc_t} each are designed to influence what strategies are used to satisfy these preferences in such a game. When all other messages \(p_{-j}\) are taken as given, \ref{oc_x} allows any individual message \(p_j\) to demand a certain level of the public good: no matter what function results from \(\sum_{i \neq j} p_i(m)\), the addition of \(p_j\) can create an integrand \(\sum_{i=1}^n p_i(m)\) such that \(\mathcal{X}(\mathbf{p})\) produces any outcome \(x\) an individual \(j\) desires. Outcome function \ref{oc_t} states that the tax schedule that will be administered to any individual does not depend on their own message. In tandem, these two outcome functions mean that a utility optimizing agent playing a game induced by this mechanism implicitly has only one margin of optimization, the level of the public good their message demands. 
 
\begin{proposition}
For quasi-concave preferences, at any Nash equilibrium induced by our mechanism, any equilibrium outcome \((\vec{\tau}, x)\) coincides with a generalized Lindahl outcome and is Pareto optimal.
\end{proposition}

\begin{proof}
To be a generalized Lindahl outcome, each individual must be given a personal tax schedule, the cost of providing the public good must be covered, and the allocation \((\vec{\tau}, x)\) must be individually rational and Pareto optimal. The budget constraint of a player is \(\int_0^x p_i(t) dt + y_i = \omega_i \) and their utility is defined by some quasi-concave function \(u_i(x, y_i)\). Each player faces the following utility maximization problem:

\begin{align*}
&\max_{p_i(x)} ~u_i \left( x, \omega_i - \int_0^x \tau_i(t) dt\right)
\\
&s.t.~ x = \argmax_{t \geq 0} \int_o^t \sum_{i=1}^n p_i(m) dm
\end{align*}

\noindent or

\begin{align*}
&\max_{p_i(x)} ~ u_i \left( x, \omega_i - \int_0^x \left[ \frac{c}{n} + p_{i+2}(t) - p_{i+1}(t) \right] dt \right) \numberthis 
\\
&s.t.~ x = \argmax_{t \geq 0} \int_o^t \sum_{i=1}^n p_i(m) dm \numberthis \label{x_cons}
\end{align*}

The problem can be simplified further, as a choice over \(x\). The objective function does not \textit{directly} depend on an individual's choice of message \(p_i\), but the amount of the public good provided does: if a player understands \(x\) is chosen according to \ref{x_cons}, then they understand that the first order condition \(\sum_{i=1}^n p_i(x) = 0\) must hold. Each player takes all other messages \(p_{-i}\) as given, and submits a \(p_i\) such that the maximizing argument of \ref{x_cons} is the \(x\) they deem optimal. Only by this level of \(x\) does \(p_i\) affect their utility; the problem can be rewritten as:

\begin{equation}
\max_x ~ u_i \left( \omega_i - \int_0^x \left[ \frac{c}{n} + p_{i+2}(t) - p_{i+1}(t) \right] dt, x \right) 
\end{equation}

\noindent The first order condition for utility maximization is: 

\begin{align*}
\frac{u_{i, x}(x, y_i(x))}{u_{i, y_i}(x, y_i(x))} &= \frac{c}{n} + p_{i+2}(x) - p_{i+1}(x) \numberthis \label{foc}
\end{align*}

\noindent Equation \ref{foc} states that in a Nash equilibrium a best responding individual submits a message \(p_i\) such that the level of the public good the message implicitly demands equates their marginal rate of substitution to their personalized tax at the margin demanded. 

The quasi-concavity of \(u_i\) in tandem with \ref{foc} establishes the individual rationality of any Nash equilibrium. If \(u_i\) is quasi-concave, then \ref{foc} defines the critical point \(x^*\) of a global maximum level of utility for each player, \(u_i(x^*, y_i(x^*))\), taking all other messages as given. It is then the best response of a player to demand \(x^*\). In any Nash equilibrium where all players are best responding, they must all demand the same \(x^*\), and so in such an equilibrium they all achieve their global maximum level of utility and it is individually rational for them to participate.

The sum of all \(n\) first order conditions is 

\begin{equation}
\sum_{i=1}^n \frac{u_{i, x}(x, y_i(x))}{u_{i, y_i}(x, y_i(x))} = c
\label{eq}
\end{equation}

\noindent The modulo dependence of the personalized taxes creates a cancellation that leads to the Samuelson marginal condition: the outcome is Pareto optimal. 

The final qualification left to prove is that the cost of provision is covered. This is true via the construction of the personalized tax outcome function. No matter what messages are submitted, the sum of all tax schedules \(\tau_i\) equals marginal cost at every single margin, \(\sum_{i=1}^n \tau_i = c\). 

Any Nash equilibrium of the game induced by a generalized Walker mechanism is a generalized Lindahl outcome: it provides individualized tax schedules, covers the cost of provision of the public good, and is Pareto optimal and individually rational. 

\end{proof}

Note that the best response message \(p_i\) in the proof above is not unique. \ref{foc} only applies at a single point, the level of the public good provided.  The only feature of \(p_i\) relevant to utility maximization is where it dictates \(\sum_{i=1}^n p_i(x) = 0\). Consequently, this is also the only feature relevant to the achievement of a Nash equilibrium. Therefore, a best response \(p_i\) could be any integrable function that satisfies the point constraint \ref{foc}. This is the result of the simple structure of our message space, and an opportunity to improve the generalized Walker mechanism in future research. We discuss this more in the following section.

\section{Discussion}


	So far we have shown that any Nash equilibrium implemented by our mechanism is a generalized Lindahl equilibrium. For completeness, we would like to prove the following as well.
	
\begin{proposition}
Any generalized Lindahl outcome can be implemented by a unique Nash equilibrium of a game induced by a generalized Walker mechanism.
\end{proposition}
	
	This would theoretically allow a social planner to create a mechanism that picks whatever generalized Lindahl outcome it desires. In order to prove proposition 2, the generalized Walker mechanism we currently have will have to be adjusted to elicit more information. Our outcome functions \ref{oc_x} and \ref{oc_t} do not provide enough structure.
		
	For example, suppose we have a generalized Lindahl outcome described by (\(\tau_1, ..., \tau_n\)) and (\(x, y_1, ..., y_n\)). If we follow Walker (1981) to attempt to prove proposition 2 (as we follow him to prove proposition 1), we  construct the following system of equations from this outcome and \ref{oc_x} and \ref{oc_t}.
	
\begin{align*}
\argmax_{t \geq 0} \int_0^t \sum_{i=1}^n p_i(m) dm &= x \\
p_2 - p_1 &= \tau_n - \frac{c}{n} \\
p_3 - p_2 &= \tau_1 - \frac{c}{n} \\
\vdots& \\
p_n - p_{n-1} &= \tau_{n-2} - \frac{c}{n} 
\end{align*}
	
If this system implicitly defines a unique set of messages (\(p_1, ...,p_n\)), then that set of messages uniquely implements the generalized Lindahl outcome described by (\(\tau_1, ..., \tau_n\)) and (\(x, y_1, ..., y_n\)). Unfortunately, this system does not have enough information to do so. The messages are functions, and so need to be defined across all margins of provision to be unique. The \(n-1\) personalized tax equations provide such definitions, but the outcome function that dictates the level of the public good is only defined on a single margin, \(x\). The only constraint it puts on the messages is that their sum must equal zero at \(x\). Consequently, this system defines an infinite number of Nash equilibria capable of achieving any single generalized Lindahl outcome. Proving proposition 2 is impossible with the mechanism as it stands. 

This issue arises from the fact that an individual's message \(p_i\) only affects their utility through the level of \(x\) it implicitly chooses. This simplification helped prove proposition 1, but it handicaps our ability to prove proposition 2. 

To correct this issue, we are considering expanding the message space to include both a function \(p_i\) and a scalar \(\chi_i\). This would allow a player's message to pick a level of the public good, via the scalar, independently of their submitted function. This additional information should provide the structure we need to carry out both proofs.
	
An additional direction to take this research is to propose a mechanism that implements stable generalized Lindahl equilibria. Such a mechanism could be supermodular and/or contractionary.  

The design of a supermodular mechanism is complicated by the requirement to generate personalized tax schedules. A necessary condition for supermodularity is that the best responses of players are strategic complements. When a best response is a scalar message, such strategic complementarities are easy to envision. Yet we need our hypothetical mechanism to elicit messages that create tax schedules. It is hard to see how a message could do so without containing a continuous function in itself. It is even less obvious how to define as a compliment a best response whose output is a continuous function. Perhaps, for example, the partial derivative of best response \(A\) at each margin over which it is defined needs to be increasing in the value of the of best response \(B\) at the same margin. 

The design of a contraction mechanism is also complicated by the necessity to create tax schedules. To understand why, we need to introduce the following definitions (Healy and Mathevet 2012). 


\begin{definition}
 if \(M\) is a space (such as \(\mathbb{R}^2\)) with metric d (such as the Euclidean norm), then a function \(f: M \rightarrow M\) is a contraction mapping if for some \(\epsilon \in (0,1)\) the following holds for all \(m, m' \in M\): 

\[
d(f(m), f(m')) \leq \epsilon  d(m, m')
\]

\end{definition}

\begin{definition}
If we have a metric space \((M, d)\) and a mechanism \(\Gamma = (M, h)\) where \(M\) is our message space, \(d\) is a metric, and \(h\) is an outcome function; and if we have a profile of player types \(\theta \in \Theta\) with preferences \(u_i( m | \theta_i)\) s.t. \(m \in M\); then \(\Gamma\) is a contraction mechanism if the game it induces with \(u_i\) has a single-valued best response function \(f(\cdot, \theta): M \rightarrow M\) that is a d-contraction mapping. 
\end{definition}

The point of complication arises in definition 3 with the message space \(M\). We need \(m \in M\) to be able to generate a tax schedule via some outcome function. It is hard to see how to do so without having \(M\) include some space of continuous functions. If, for example, we choose to work with a message space \(M = \mathbb{C}(\mathbb{R})^n\), we would need to create a distance metric that somehow quantifies the distance between a pair of continuous functions. Such a distance metric may make it very diffucult to prove a mechanism is contractive. 



\section{Conclusion}

\section{Appendix}

The solution will take place as follows: I will solve for \(p_1\) in terms of \(p_2, v_1\) and \(v_2\), and then substitute \(p_1\) into \ref{iso_simp}; I will iterate this process two more times, substituting equations for \(p_2\) and \(p_3\) into \ref{iso_simp}; at that point a pattern will emerge that will allow me to extrapolate a closed-form solution for to \(p_n\), and then for \(p_{n-1}\); finally, I will use the pattern observable in the solutions for \(p_n\) and \(p_{n-1}\) solution to deduce a closed form solution for \(p_i\) in general.

As stated, first solve for \(p_1\) 
	
\[
p_1 = \frac{-v_2}{\Omega_{21}} + v_1 + \frac{p_2}{\Omega_{21}}
\]

\noindent Substitute into \ref{iso_simp}

\[
\left( \frac{-v_2}{\Omega_{21}} + v_1 + \frac{p_2}{\Omega_{21}} \right) + p_2 + p_3 + ... + p_n = c
\]

Combine terms, keeping prices and marginal benefits organized, and note that \(\Omega_{ij} = \Omega_{ji}^{-1}\). We get our first instance of a pattern.

\[
\left( 1 + \Omega_{12} \right) p_2 + p_3 + ... + p_n + v_1 - \Omega_{12} v_2 = c
\]

Iterating, solve for \(p_2\) in terms of \(p_3, v_2\) and \(v_3\) using the equation relating \(p_2\) to \(p_3\) from our system of equalities. Substitute into above.

\[
\left( 1 + \Omega_{12} \right) \left( \frac{-v_3}{\Omega_{32}} + v_2 + \frac{p_3}{\Omega_{32}} \right) + p_3 + p_4 + ... + p_n + v_1 -\Omega_{12} v_2 = c
\]

Combine terms again, and note that \(\Omega_{Ki} / \Omega_{Kj} = \Omega_{ij}\). A nice cancellation occurs. We get a second instance of the pattern.

\[
\left( 1 + \Omega_{23} + \Omega_{13} \right) p_3 + p_4 + ... + p_n + v_1 + v_2 - (\Omega_{23} + \Omega_{13}) v_3 = c
\]

Iterate one last time, substituting for \(p_3\).

\[
\left( 1 + \Omega_{23} + \Omega_{13} \right) \left(  \frac{-v_4}{\Omega_{43}} + v_3 + \frac{p_4}{\Omega_{43}} \right) + p_4 + p_5 + ... + p_n + v_1 + v_2 - (\Omega_{23} + \Omega_{13}) v_3 = c
\]

Combine terms, and we get a third instance of the pattern.

\[
\left( 1 + \Omega_{34} + \Omega_{24} + \Omega_{14} \right) p_4 + p_5 + ... + p_n + v_1 + v_2 + v_3 - (\Omega_{34} + \Omega_{24} + \Omega_{14}) v_4 = c
\]

Let's quickly look at all three instances of the pattern together.

\begin{align*}
&\left( 1 + \Omega_{12} \right) p_2 + p_3 + ... + p_n + v_1 - \Omega_{12} v_2 = c
\\
&\left( 1 + \Omega_{23} + \Omega_{13} \right) p_3 + p_4 + ... + p_n + v_1 + v_2 - (\Omega_{23} + \Omega_{13}) v_3 = c
\\
&\left( 1 + \Omega_{34} + \Omega_{24} + \Omega_{14} \right) p_4 + p_5 + ... + p_n + v_1 + v_2 + v_3 - (\Omega_{34} + \Omega_{24} + \Omega_{14}) v_4 = c
\end{align*}

By extrapolation, and noting that \(\Omega_{ii} = 1\), \(p_n\) must be implicitly defined by

\[
p_n \sum_{i=1}^n \Omega_{in} + \sum_{i=1}^{n-1} v_i - v_n \sum_{i=1}^{n-1} \Omega_{in} = c
\]

Solving for \(p_n\), we get

\begin{equation}
p_n = \left( \sum_{i=1}^n \Omega_{ni} \right)^{-1} \left[  c - \sum_{i=1}^{n-1} v_i + v_n \sum_{i=1}^{n-1} \Omega_{ni}\right]
\end{equation}

To get a closed form solution for any \(p_i\), I'll first use 6 to solve for \(p_{n-1}\) using the equation from our system of equations that relates \(p_{n-1}\) to \(p_n\). Then with closed from solutions for both \(p_n\) and \(p_{n-1}\), a pattern emerges that I use to solve for the general closed form solution for \(p_i\).

First, solve for \(p_{n-1}\). We know that

\[
p_{n-1} = \frac{-v_n}{\Omega_{n,n-1}} + v_{n-1} + \frac{p_n}{\Omega_{n,n-1}}
\]

Substitute in 6. Step by step simplification:

\begin{align*}
p_{n-1} &= \frac{-v_n}{\Omega_{n,n-1}} + v_{n-1} + \frac{1}{\Omega_{n,n-1}} \left( \sum_{i=1}^n \Omega_{in} \right)^{-1} \left[  c - \sum_{i=1}^{n-1} v_i + v_n \sum_{i=1}^{n-1} \Omega_{in}\right]
\\
& =  - \Omega_{n-1,n} v_n + v_{n-1} + \left( \sum_{i=1}^n \Omega_{n,n-1} \Omega_{in} \right)^{-1} \left[  c - \sum_{i=1}^{n-1} v_i + v_n \sum_{i=1}^{n-1} \Omega_{in}\right]
\\
& = - \Omega_{n-1,n} v_n + v_{n-1} + \left( \sum_{i=1}^n \Omega_{i, n-1} \right)^{-1} \left[  c - \sum_{i=1}^{n-1} v_i + v_n \sum_{i=1}^{n-1} \Omega_{in}\right]
\\
& =  \left( \sum_{i=1}^n \Omega_{i, n-1} \right)^{-1} \left[ - \Omega_{n-1,n} v_n \sum_{i=1}^n \Omega_{i, n-1} + v_{n-1} \sum_{i=1}^n \Omega_{i, n-1} + c - \sum_{i=1}^{n-1} v_i + v_n \sum_{i=1}^{n-1} \Omega_{in} \right]
\\
& =  \left( \sum_{i=1}^n \Omega_{i, n-1} \right)^{-1} \left[ - v_n \sum_{i=1}^n \Omega_{in} + v_{n-1} \sum_{i=1}^n \Omega_{i, n-1} + c - \sum_{i=1}^{n-1} v_i + v_n \sum_{i=1}^{n-1} \Omega_{in} \right]
\\
&  =  \left( \sum_{i=1}^n \Omega_{i, n-1} \right)^{-1} \left[ - v_n \sum_{i=1}^n \Omega_{in} + v_{n-1} \sum_{i=1}^n \Omega_{i, n-1} + c - \sum_{i=1}^{n-1} v_i + v_n \sum_{i=1}^{n-1} \Omega_{in} \right]
\\
& =  \left( \sum_{i=1}^n \Omega_{i, n-1} \right)^{-1} \left[ v_{n-1} \sum_{i=1}^n \Omega_{i, n-1} + c - \sum_{i=1}^{n-1} v_i + v_n \left( \sum_{i=1}^{n-1} \Omega_{in} - \sum_{i=1}^n \Omega_{in} \right) \right]
\end{align*}

At this point, we need to break steps down term by term. Note that the term in the parenthesis within the brackets equals -1, and that

\[
v_{n-1} \sum_{i=1}^n \Omega_{i, n-1} = v_{n-1} \left( 1 +\sum_{i \neq n-1} \Omega_{i, n-1} \right) =  v_{n-1} + v_{n-1} \sum_{i \neq n-1} \Omega_{i, n-1}
\]

 The lone \(v_{n-1}\) will cancel with the final term in the \( \sum_{i=1}^{n-1} v_i \) summation. This leaves us with
 
 \[
 p_{n-1} =  \left( \sum_{i=1}^n \Omega_{i, n-1} \right)^{-1} \left[ v_{n-1} \sum_{i \neq n-1} \Omega_{i, n-1} + c - \sum_{i=1}^{n-2} v_i - v_n \right]
 \]
 
 Or
 
 \begin{equation}
 p_{n-1} =  \left( \sum_{i=1}^n \Omega_{i, n-1} \right)^{-1} \left[ c - \sum_{i \neq n-1} v_i + v_{n-1} \sum_{i \neq n-1} \Omega_{i, n-1}  \right]
 \end{equation}

Our closed form solution to \(p_{n-1}\). Finally let's compare 7 to 6.

\begin{align*}
 &p_{n-1} =  \left( \sum_{i=1}^n \Omega_{i, n-1} \right)^{-1} \left[ c - \sum_{i \neq n-1} v_i + v_{n-1} \sum_{i \neq n-1} \Omega_{i, n-1}  \right]
\\
&p_n = \left( \sum_{i=1}^n \Omega_{in} \right)^{-1} \left[  c - \sum_{i=1}^{n-1} v_i + v_n \sum_{i=1}^{n-1} \Omega_{in}\right]
\end{align*}

From the above pattern, it's clear that the closed form solution for any \(p_j\) is

\begin{equation}
p_j =  \left( \sum_{i=1}^n \left( \frac{\beta_i}{\beta_j}\right)^\sigma \right)^{-1} \left[ c - \sum_{i \neq j} v_i + v_j \sum_{i \neq j} \left( \frac{\beta_i}{\beta_j}\right)^\sigma  \right]
\end{equation}

Inside the brackets, add \(0 = v_j \left( \frac{\beta_j}{\beta_j}\right)^\sigma - v_j\) and rearrange to get:

\begin{equation}
p_j =  v_j - \frac{\beta_j\strut^\sigma}{\sum_{i=1}^n \beta_i\strut^\sigma} \left[  \sum_{i =1}^n (v_i)  - c \right]
\end{equation}

\begin{thebibliography}{1}

\bibitem{} Chen, Yan. "A family of supermodular Nash mechanisms implementing Lindahl allocations." Economic Theory 19.4 (2002): 773-790.

\bibitem{} Diamantaras, Dimitrios, and Robert P. Gilles. "The pure theory of public goods: efficiency, decentralization, and the core." International Economic Review (1996): 851-860.

\bibitem{} Foley, Duncan K. "Lindahl's Solution and the Core of an Economy with Public Goods." Econometrica: Journal of the Econometric Society (1970): 66-72.

\bibitem{} Groves, Theodore, and John Ledyard. "Optimal allocation of public goods: A solution to the" free rider" problem." Econometrica: Journal of the Econometric Society (1977): 783-809.

\bibitem{} Healy, Paul J., and Laurent Mathevet. "Designing stable mechanisms for economic environments." Theoretical economics 7.3 (2012): 609-661.

\bibitem{} Hurwicz, Leonid. "Outcome functions yielding Walrasian and Lindahl allocations at Nash equilibrium points." The Review of Economic Studies 46.2 (1979): 217-225.

\bibitem{} Kaneko, Mamoru. "The ratio equilibrium and a voting game in a public goods economy." Journal of Economic Theory 16.2 (1977): 123-136.

\bibitem{} Lindahl, Erik. "Just taxation—a positive solution." Classics in the theory of public finance. Palgrave Macmillan, London, 1958. 168-176.

\bibitem{} Mas-Colell, Andreu. "Efficiency and decentralization in the pure theory of public goods." The Quarterly Journal of Economics 94.4 (1980): 625-641.

\bibitem{} Mathevet, Laurent. "Supermodular mechanism design." Theoretical Economics 5.3 (2010): 403-443.

\bibitem{} Mas-Colell, Andreu, and Joaquim Silvestre. "Cost share equilibria: A Lindahlian approach." Journal of Economic Theory 47.2 (1989): 239-256.

\bibitem{} Milgrom, Paul, and John Roberts. "Rationalizability, learning, and equilibrium in games with strategic complementarities." Econometrica: Journal of the Econometric Society (1990): 1255-1277.

\bibitem{} Van Essen, Matthew J. "A simple supermodular mechanism that implements Lindahl allocations." Journal of Public Economic Theory 15.3 (2013): 363-377.

\bibitem{} Varian, Hal R. "A solution to the problem of externalities when agents are well-informed." The American Economic Review (1994): 1278-1293.

\bibitem{} Varian, Hal R. "Sequential contributions to public goods." Journal of Public Economics 53.2 (1994): 165-186.

\bibitem{} Walker, Mark. "A simple incentive compatible scheme for attaining Lindahl allocations." Econometrica: Journal of the Econometric Society (1981): 65-71.

\bibitem{} Wicksell, Knut. "A new principle of just taxation." Classics in the theory of public finance. Palgrave Macmillan, London, 1958. 72-118.


\end{thebibliography}

\end{document}